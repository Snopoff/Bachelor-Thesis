\begin{titlepage} 
		\begin{center}
			\normalsize
			МИНОБРНАУКИ РОССИИ \\
			ФЕДЕРАЛЬНОЕ  ГОСУДАРСТВЕННОЕ  БЮДЖЕТНОЕ \\
			ОБРАЗОВАТЕЛЬНОЕ  УЧРЕЖДЕНИЕ  ВЫСШЕГО  ОБРАЗОВАНИЯ \\
			«ВОРОНЕЖСКИЙ  ГОСУДАРСТВЕННЫЙ  УНИВЕРСИТЕТ» \\
			(ФГБОУ ВО «ВГУ») \\[5mm]
			Факультет прикладной математики, информатики и механики\\[5mm]
			
			Кафедра вычислительной математики\\ и прикладных информационных технологий
			\vfill
			
			\textbf{Применение персистентных гомологий для задачи классификации изображений}\\[5mm]
			
			
			\bigskip
			Бакалаврская работа \\
			Направление 01.03.02 Прикладная математика и информатика\\
			Профиль Математическое моделирование и вычислительная математика

		\end{center}
		\vfill
		\newlength{\ML}
		\settowidth{\ML}{«\underline{\hspace{0.7cm}}» \underline{\hspace{2cm}}}
		\begin{minipage}{\textwidth}
			\raggedright
			Зав. кафедрой $\rule{2cm}{0.15mm}$ д.т.н., проф. Т.\,М.~Леденева  \\
			Обучающийся $\,\rule{2cm}{0.15mm}$$\rule{2.85cm}{0.0mm}$ П.\,С.~Снопов \\
			Руководитель  $\;\rule{2cm}{0.15mm}$ д.т.н., проф. Т.\,М.~Леденева
		\end{minipage}%

		\bigskip
		\vfill
		\begin{center}
			Воронеж 2021
		\end{center}
	\end{titlepage}