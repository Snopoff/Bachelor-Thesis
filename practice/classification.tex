\section{Классификация изображений датасета MNIST}

Теперь решим задачу классификации датасета MNIST. Опишем общий алгоритм \ref{classification-pipeline}, который лежит в основе решения задачи.

\medskip
\begin{algorithm}[H]
	\small
	\SetAlgoLined
	\KwData{Набор $28\text{x}28$ изображений рукописных цифр}
	\KwResult{Алгоритм, который определяет, какая цифра написана на изображении}
	\ForEach{$28\text{x}28$ изображение из выборки}{
		Построить фильтрацию;
		
		Посчитать персистентные гомологии, построить диаграмму персистентости;
		
		Векторизовать диаграмму персистентности;
	}
	
	На наборе векторизованных диаграмм обучить модель машинного обучения;
	
	
	\caption{Общий алгоритм решения задачи классификации датасета MNIST}
	\label{classification-pipeline}
\end{algorithm}
\medskip