\section{Сравнительный анализ пакетов для вычисления устойчивых гомологий}

В настоящее время существует ряд пакетов для вычисления устойчивых гомологий. Поэтому прежде чем приступать к задаче классификации, необходимо провести анализ и выбрать какой-то определенный пакет. Сравнительный анализ будет проводиться для пакетов с интерфейсом на Python, а конкретно, Dionysus \cite{dionysus}, Giotto-TDA \cite{giotto}, GUDHI \cite{gudhi}, Ripser \cite{Ripser}, которая является частью более обширной библиотеки Scikit-TDA \cite{scikit} для топологического анализа данных. 

Пакеты Dionysus и GUDHI существуют уже длительное время и поэтому более надежны в работе, когда как Scikit-TDA и Giotto-TDA являются более новыми, но быстро развивающимися библиотеками.
Сравнительный анализ этих пакетов представлен в табл. \ref{tabl:packages}.
Следующие алгоритмы используются для вычисления устойчивых гомологий:
\begin{itemize}
	\item стандартный алгоритм \cite{Zomorodian}, заключающийся в приведении граничной матрицы фильтрации к ступенчатому виду;
	\item twist algorithm \cite{twist}, заключающийся в оптимизации стандартного алгоритма путем уменьшения размерности граничной матрицы фильтрации;
	\item dual algorithm, заключающийся в вычислении устойчивых когомологий, что вычислительно более эффективно;
	\item multifield algorithm \cite{multifield}, заключающийся в использовании других полей коэффециентов для более эффективного вычисления;
	\item zigzag algorithm \cite{zigzag}, также основанный на персистентных когомологиях, которые вычисляются более эффективно. 
\end{itemize}
\begin{table}[!htbp]
	\centering
	\small
	\caption{Сравнительный анализ возможностей пакетов}	
	\begin{tabularx}{\linewidth}{|X|X|X|X|X|}
		\hline
		& Giotto-TDA & GUDHI & Ripser& Dionysus\\ \hline
		Алгоритмы для вычисления  устойчивых гомологий & standard, twist, dual& standard, dual, multifield & standard, twist, dual & standard, dual, zigzag \\ \hline
		Коэффициенты & $\mathbb{F}_p$ & $\mathbb{F}_p$ & $\mathbb{F}_p$ & $\mathbb{F}_p$ (dual);  $\mathbb{F}_2$ (standard, zigzag) \\ \hline
		Гомологии & Симплициальные, кубические& Симплициальные, кубические & Симплициальные & Симплициальные \\ \hline
		Фильтрации  & Виеторис-Рипс, Чех, $\alpha$ & $\alpha$, Виеторис-Рипс, Чех, кубические, нерв, Witness, Делоне& Виеторис-Рипс  & Виеторис-Рипс, $\alpha$, Чех\\ \hline
		Визуализация & Диаграммы устойчивости, & Диаграммы устойчивости, баркоды &Диаграмма устойчивости & Диаграммы устойчивости, баркоды \\ \hline
	\end{tabularx}
\label{tabl:packages}
\end{table}

Из таблицы видно, что библиотека GUDHI представляет максимально различные способы построения фильтраций, а также различные алгоритмы для вычисления устойчивых гомологий. Эта библиотека предоставляет также возможность вычислять кубические гомологии.

Проведем анализ пакетов на синтетических тестах. Для этого воспользуемся пакетов Tadasets \cite{Tadasets}, который, как и Ripser, является частью библиотеки Scikit-TDA \cite{scikit} и предоставляет набор синтетических датасетов, полезных для топологического анализа данных. Рассматриваемые синтетические датасеты -- это не зашумленные облака точек в $\mathbb{R}^{26}$; количество точек $n=2000$. Данные описывают следующие многообразия:
\begin{itemize}
	\item тор (расстояние от центра центра до центра внутренней окружности $c=2$, радиусX внутренней окружности $a=1$);
	\item швейцарский рулет (длина рулета $r=4$);
	\item 2-Сфера (радиус $r=3.14$);
	\item 3-Сфера (радиус $r=3.14$);
	\item букет 2 окружностей ("знак бесконечности").
\end{itemize}

Во всех пакетах использовался стандартный алгоритм вычисления персистентных гомологий. Целью данного сравнения было нахождение пакета, который быстрее других вычисляет устойчивые гомологии, и для которого потребуются минимальные настройки. Все вычисления производились в среде Google Colab \cite{colab}.
\begin{table}[!htbp]
	\centering
	\small
	\caption{Тест скорости работы пакетов}
	\begin{tabularx}{\linewidth}{|X|X|X|X|X|X|}
		\hline
		& \multicolumn{5}{c|}{Wall-time sec.} \\ \hline
		& Тор & Швейцарский рулет & 2-Сфера & 3-Сфера & "Знак бесконечности" \\ \hline
		Ripser & $26.3$ & $392$ & $6.84$ & $13.5$ & $50.4$ \\ \hline
		Gudhi & $2.2$ & $2.21$ & $2.27$ & $2.27$ & $1.51$ \\ \hline
		Giotto-TDA & $14$ & $264$ & $4.14$ & $7.04$ & $29.1$ \\ \hline
		& \multicolumn{5}{c|}{CPU time sec.} \\ \hline
		& Тор & Швейцарский рулет & 2-Сфера & 3-Сфера & "Знак бесконечности" \\ \hline
		Ripser & $22.4$ & $380$ & $6.2$ & $11.7$ & $50.4$ \\ \hline
		Gudhi & $2.12$ & $2.15$ & $2.21$ & $2.23$ & $1.45$ \\ \hline
		Giotto-TDA & $13.8$ & $263$ & $4.07$ & $6.72$ & $28.9$ \\ \hline
	\end{tabularx}	
	\label{tabl:tests}
\end{table}

На основе данных сравнений можно сделать вывод, что библиотека GUDHI предоставляет наиболее эффективный способ вычисления устойчивых гомологий. В свою очередь, библиотека Dionysus тратила очень много времени на построение фильтрации, поэтому в вычислительном эксперименте ее результатов нет.

Таким образом, библиотека GUDHI представляет наибольшее разнообразие методов построения фильтраций, алгоритмов вычисления устойчивых гомологий, а также является эффективной с вычислительной точки зрения. Однако, в дальнейшем будет использоваться Giotto-TDA, так как, несмотря на то, что в ряде тестов эта библиотека оказалась не самой эффективной с вычислительной точки зрения (например, на датасете "Швейцарский рулет"\ Giotto-TDA показала себя достаточно неэффективно, хотя все еще лучше, чем Ripser), эта библиотека предоставляет наиболее удобный интерфейс для использования, так как сам пакет сделан на основе пакета Scikit-Learn, одного из основных пакетов для машинного обучения на Python, а потому все методы имеют похожую структуру, какую имеют методы из пакета Scikit-Learn и других пакетов для научных вычислений на Python (например, NumPy).
