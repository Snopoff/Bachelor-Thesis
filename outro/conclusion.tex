\chapter*{Заключение}
\addcontentsline{toc}{chapter}{Заключение}
В ходе данной работы был изучен теоретический материал по алгебраической и прикладной топологии и машинному обучению. Был реализован алгоритм классификации датасета MNIST, используя только топологические свойства изображенных рукописных цифр, который выдавал высокий уровень точности.

В ходе данной работы было выявлено, что реализованный алгоритм является эффективным алгоритмом понижения размерности пространства признаков. Было проведено сравнение с другими моделями машинного обучения, которые были обучены на обычных признаках -- плоских представлениях изображений. В результате данного сравнения было обнаружено, что реализованный алгоритм показывает более высокую точность классификации при меньшем числе признаков.
