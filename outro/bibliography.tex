\begin{thebibliography}{99}
	\bibitem{Viro}
	Виро О.Я., Иванов О.А., Нецветаев Н.Ю., Харламов В.М. (2018). Элементарная топология. 978-5-4439-2680-3.
	\bibitem{Vick}
	Вик Дж. У. (2005). Теория гомологий. Введение в алгебраическую топологию. 5-94057-086-0.
	\bibitem{Hatcher}
	Хатчер А. (2011). Алгебраическая топология. 978-5-94057-748-5.
	\bibitem{Munkres}
	Munkres, James R. (2000). Topology. 0-13-181629-2.
	\bibitem{MathAndMyth}
	Фоменко А.Т. (2001). Математика и миф сквозь призму геометрии. URL: \url{http://dfgm.math.msu.su/files/fomenko/myth-vved.php}
	\bibitem{Edelsbrunner et al}
	Edelsbrunner, H., Letscher, D., and Zomorodian, A. (2002). Topological persistence and simplification. Discrete Comput. Geom., 28:511–533.
	\bibitem{Carlsson and Zomorodian}
	Zomorodian, A. and Carlsson, G. (2005). Computing persistent homology. Discrete Comput.
	Geom., 33(2):249–274.
	\bibitem{Carlsson}
	Carlsson, G. (2009). Topology and data. AMS Bulletin, 46(2):255–308.
	\bibitem{Fasy et al}
	Brittany T. Fasy, Jisu Kim, Fabrizio Lecci, Clement
	Maria, David L. Millman, and Vincent Rouvreau. Introduction to the R package TDA. URL: \url{https://cran.r-project.org/web/packages/TDA/vignettes/article.pdf}
		\bibitem{base}
	F. Chazal An introduction to Topological Data Analysis: fundamental and practical aspects for data scientists / F.Chazal, B. Michel. URL : \url{https://arxiv.org/pdf/1710.04019.pdf}
	
	\bibitem{Edelsbrunner}
	H. Edelsbrunner Computational Topology An Introduction / H. Edelsbrunner, J. Harer ; AMS : Providence, 2009. -- 241 с.
	
	\bibitem{Zomorodian} 
	A. Zomorodian Computing persistent homology / A. Zomorodian, G. Carlsson // Discrete Comput. Geom. -- 2005. -- Vol. 33, № 2. -- P. 249 -- 274. URL : \url{https://geometry.stanford.edu/papers/zc-cph-05/zc-cph-05.pdf}
	
	\bibitem{alsobase}
	N. Otter A roadmap for the computation of persistent homology / N. Otter, M.A. Porter, U. Tillmann, P. Grindrod, H. A Harrington // EPJ Data Sci. -- 2017. -- Vol. 6, №17. URL : \url{https://epjdatascience.springeropen.com/articles/10.1140/epjds/s13688-017-0109-5}
	
	\bibitem{giotto}
	Giotto-tda -- библиотека для топологического анализа данных на Python. -- URL : \url{https://github.com/giotto-ai/giotto-tda}
	
	\bibitem{gudhi}
	GUDHI -- библиотека для топологического анализа данных на C++ с интерфейсом для Python. -- URL : \url{https://gudhi.inria.fr/}
	
	\bibitem{modules}
	F. Chazal The Structure and Stability of Persistence Modules / F. Chazal, V. de Silva, M. Glisse, S. Outdot. URL : \url{https://arxiv.org/pdf/1207.3674.pdf}
	
	\bibitem{dionysus}
	Dionysus 2 -- библиотека для вычислений устойчивых гомологий, написанная на C++ и имеющая интерфейс на Python. URL : \url{https://www.mrzv.org/software/dionysus/}
	
	\bibitem{Tadasets}
	Tadasets -- библиотека, содержащая синтетические датасеты для топологического анализа данных. URL : \url{https://github.com/scikit-tda/tadasets}
	
	\bibitem{Ripser}
	Интерфейс для Ripser на Python. URL : \url{https://ripser.scikit-tda.org/en/latest/}
	
	\bibitem{scikit}
	Scikit-TDA -- библиотека для топологического анализа данных, содержащая Ripser и Tadasets. URL : \url{https://scikit-tda.org/}
	
	\bibitem{colab}
	Google Colab -- Сервис Google, предоставляющий возможность запускать код, написанный на Python, в браузере, обладающий интерфейсом Jupyter Notebook, специально созданный для задач машинного обучения, анализа данных, а также образования. URL : \url{https://colab.research.google.com/}
	
	\bibitem{twist}
	C. Chen Persistent Homology Computation with a Twist / C. Chen, M. Kerber // EuroCG -- 2011. URL : \url{https://eurocg11.inf.ethz.ch/abstracts/22.pdf}
	
	\bibitem{multifield}
	J. Boissonnat Computing Persistent Homology with Various Coefficient Fields in a Single Pass / J.Boissonnat, C. Maria. URL : \url{https://arxiv.org/abs/2001.02960}.
	
	\bibitem{zigzag}
	C. Maria Computing Persistent Cohomology / C. Maria, S. Oudot. URL: \url{https://arxiv.org/abs/1608.06039}.
	
\end{thebibliography}