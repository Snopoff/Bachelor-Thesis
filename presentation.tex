\documentclass{beamer}
\usepackage[english,russian]{babel}   % руссификации AmSLaTeX
\usetheme{CambridgeUS}

\usepackage{amsmath,amsfonts,amssymb,euscript,graphicx,wrapfig,multirow,mathtools,amsthm}
\usepackage{dsfont}

%%% Работа с картинками
\usepackage{graphicx}  % Для вставки рисунков
\graphicspath{{./images/}}  % папки с картинками
\setlength\fboxsep{3pt} % Отступ рамки \fbox{} от рисунка
\setlength\fboxrule{1pt} % Толщина линий рамки \fbox{}
\usepackage{wrapfig} % Обтекание рисунков и таблиц текстом
\usepackage[export]{adjustbox}
\usepackage{caption}
\usepackage{subcaption}
\usepackage{float}
\usepackage{tikz-cd}
\usetikzlibrary{babel}
\usepackage{animate}
\usepackage{tabularx}

%Algorithms
%http://blog.harrix.org/article/648
\usepackage[ruled,vlined]{algorithm2e}

\SetKwInput{KwData}{Исходные параметры}
\SetKwInput{KwResult}{Результат}
\SetKwInput{KwIn}{Входные данные}
\SetKwInput{KwOut}{Выходные данные}
\SetKwIF{If}{ElseIf}{Else}{если}{тогда}{иначе если}{иначе}{конец условия}
\SetKwFor{While}{до тех пор, пока}{выполнять}{конец цикла}
\SetKw{KwTo}{от}
\SetKw{KwRet}{возвратить}
\SetKw{Return}{возвратить}
\SetKwBlock{Begin}{начало блока}{конец блока}
\SetKwSwitch{Switch}{Case}{Other}{Проверить значение}{и выполнить}{вариант}{в противном случае}{конец варианта}{конец проверки значений}
\SetKwFor{For}{цикл}{выполнять}{конец цикла}
\SetKwFor{ForEach}{для каждого}{выполнять}{конец цикла}
\SetKwRepeat{Repeat}{повторять}{до тех пор, пока}
\SetAlgorithmName{Алгоритм}{алгоритм}{Список алгоритмов}

\DeclareMathOperator{\im}{im}
\DeclarePairedDelimiter\norm{\lVert}{\rVert}%
\newcommand{\quotient}[2]{{\raisebox{.2em}{$#1$}\left/\raisebox{-.2em}{$#2$}\right.}}

\makeatother
\setbeamertemplate{footline}
{
	\leavevmode%
	\hbox{%
		\begin{beamercolorbox}[wd=.8\paperwidth,ht=2.25ex,dp=1ex,center]{title in head/foot}%
			\usebeamerfont{title in head/foot}\insertshorttitle
		\end{beamercolorbox}%
		\begin{beamercolorbox}[wd=.2\paperwidth,ht=2.25ex,dp=1ex,center]{date in head/foot}%
			\insertframenumber{} / \inserttotalframenumber\hspace*{1ex}
	\end{beamercolorbox}}%
	\vskip0pt%
}
\makeatletter

\title{Применение персистентных гомологий для задачи классификации изображений}

\author{\small Обучающийся \qquad\qquad\qquad\qquad\qquad Снопов П.М.\\
		\enspace Руководитель \qquad  д.т.н., проф. \qquad Леденева Т.М.}
\institute[ВГУ]{{ Воронежский Государственный Университет} \\ 
				Факультет прикладной математики, информатики и механики \\[1em]
				Кафедра вычислительной математики и прикладных \\
				информационных технологий}

\date{\footnotesize Бакалаврская работа \\
	  Направление 01.03.02 Прикладная математика и информатика \\
  	  Профиль Математическое моделирование и вычислительная математика}


\begin{document}
	\begin{frame}[plain]
		\centering
		
		\begin{beamercolorbox}[sep=8pt,center,colsep=-4bp,rounded=true,shadow=true]{institute}
			\usebeamerfont{institute}\insertinstitute
		\end{beamercolorbox}
		
		{\usebeamercolor[fg]{titlegraphic}\inserttitlegraphic\par}
		
		\begin{beamercolorbox}[sep=8pt,center,colsep=-4bp,rounded=true,shadow=true]{title}
			\usebeamerfont{title}\inserttitle\par%
			\ifx\insertsubtitle\@empty%
			\else%
			\vskip0.25em%
			{\usebeamerfont{subtitle}\usebeamercolor[fg]{subtitle}\insertsubtitle\par}%
			\fi%     
		\end{beamercolorbox}%
		
		\vskip1em\par
		
		\begin{beamercolorbox}[sep=8pt,center,colsep=-4bp,rounded=true,shadow=true]{date}
			\usebeamerfont{date}\insertdate
		\end{beamercolorbox}\vskip0.5em
		
		\begin{beamercolorbox}[sep=8pt,center,colsep=-4bp,rounded=true,shadow=true]{author}
			\usebeamerfont{author}\insertauthor
		\end{beamercolorbox}
		
	\end{frame}
	\footnotesize
	\section{Введение}
		\subsection{Содержание}	
	\begin{frame}
		\frametitle{Содержание}
		\tableofcontents
	\end{frame}
		\subsection{Цель и задачи работы}
		\begin{frame}
		\frametitle{Цель и задачи работы}
			\textbf{\small Цель:} Исследование подхода, основанного на топологическом анализе данных, для классификации изображений \\[1em]
			\textbf{\small Задачи:}
			\begin{itemize}
				\item Изучение теоретических и практических основ топологического анализа данных
				\item Анализ подходов классификации изображений
				\item Формирование алгоритма на основе персистентных гомологий
				\item Проведение вычислительного эксперимента, выявление области применимости, плюсов и минусов данного подхода
			\end{itemize}
		\end{frame}
		\subsection{Актуальность}	
		\begin{frame}
		\frametitle{Актуальность}
			\begin{itemize}
				\item Классификация изображений -- фундаментальная задача анализа данных, одно из основных направлений ее развития, которое в последнее время имеет очень интенсивное развитие, связанное с достижениями нейронных сетей.
				\item Основным инструментом ТДА являются персистентные гомологии, о которых можно думать как об адаптации понятия гомологии к облаку точек. С помощью такого инструмента можно выявлять топологические характеристики исследуемого объекта.
				\item Задача классификации по сути является задачей определения характеристических свойств, которым удовлетворяют объекты одного класса. Поэтому персистентные гомологии могут быть полезны в данной задаче, так как зачастую такие свойства имеют геометрическую природу.
			\end{itemize}
	\end{frame}

	\section{Теоретические сведения}
	
		\subsection{Симплициальные комплексы}
		\begin{frame}
			\frametitle{Симплициальные комплексы}
			
			\begin{definition}
				Симплициальный комплекс $K$ -- это множество симплексов, т.е. выпуклых оболочек набора $n+1$ точек $\in \mathbb{R}^p$, таких, что векторы $ x_1 - x_0, ..., x_n - x_0 $ линейно независимы, при этом
				\begin{itemize}
					\setlength{\itemsep}{-1mm}
					\item Для каждого симплекса из $K$ его грани тоже лежат в $K$,
					\item Пересечение любых двух симплексов $\sigma, \tau \in K$ либо пусто, либо является гранью и $\sigma$, и $\tau$.
				\end{itemize}
			\end{definition}
			\begin{figure}
				\centering
				\includegraphics[width=\linewidth]{simplexAndComplex.png}
			\end{figure}
		\end{frame}
		\subsection{Cимплициальные гомологии}
		\begin{frame}[fragile]
			\frametitle{Симплициальные гомологии}
			Для симплициального комплекса можно посчитать его группы симплициальных гомологий $H_n$. Ранк $n$-ой группы -- это $n$-ое число Бетти $\beta_n$. Оно отражает количество $n$-мерных особенностей комплекса. Для $n=0$, $\beta_0$ отражает количество компонент связности данного пространства. При $n=1$ -- количество циклов. При $n=2$ число Бетти $\beta_2$ описывает количество "полостей".
			
			\begin{figure}[!htbp]
				\centering
				\includegraphics[width=0.5\linewidth, keepaspectratio=true]{betti_numbers.png}
			\end{figure}
		\end{frame}
		
		\subsection{Фильтрации и устойчивые гомологии}
		\begin{frame}
			\frametitle{Как построить симплициальный комплекс?}
			
			Имея облако точек $X \subset \mathbb{R}^n$, можно построить симплициальный комплекс по нему, например комплекс \text{\it Вьеториса—Рипса}.
			\begin{figure}
				\centering
				\includegraphics[scale=0.3]{vr.png}
			\end{figure}
		\end{frame}
		\begin{frame}
			\frametitle{Фильтрации и устойчивые гомологии}
			{\it Фильтрацией симплициального комплекса $K$} называют вложенное семейство подкомплексов $ (K_\tau)_{\tau \in T} $, такое, что если $ \tau < \tau^{'} $, то $ K_\tau \subseteq K_{\tau^{'}} $.
			$n$-ыми устойчивыми гомологиями будем называть семейство $n$-гомологий для каждого члена фильтрации. Естественный порядок на фильтрации будет порождать естественные гомоморфизмы между группами гомологий. Таким образом, устойчивые гомологии отслеживают появление и исчезновение топологических особенностей в фильтрации.
			\begin{figure}
				\centering
				\includegraphics[width=\linewidth]{filtration.png}
		\end{figure}
		\end{frame}
		\begin{frame}
			\frametitle{Фильтрации и устойчивые гомологии}
			\begin{figure}
				\centering
				\begin{subfigure}{.45\textwidth}
					\centering
					\includegraphics[scale=0.5, width=\linewidth]{persist_diag.png}
					\caption{Диаграмма персистентности}
				\end{subfigure}
				\begin{subfigure}{.45\textwidth}
					\centering
					\includegraphics[width=\linewidth]{barcode.png}
					\caption{Баркод}
				\end{subfigure}
				\caption{Способы кодирования информации об персистентных гомологиях}
			\end{figure}
		\end{frame}
		\subsection{Векторизация диаграмм персистентности}
		\begin{frame}
			\frametitle{Метрическое пространство всех диаграмм персистентности}
			На множестве $\mathcal{D}$ диаграмм персистентности можно ввести структуру метрического пространства, однако для большинства алгоритмов машинного обучения и анализа данных требуется более сложная структура.
			
			Можно векторизовать диаграмму, т.е. построить отображение $ \varphi: \mathcal{D} \to V $, где $V$ -- нормированное векторное пространство. Тогда диаграмме $B$ можно сопоставить число -- $\norm{\varphi(B)}$.
			\begin{figure}[!htbp]
				\centering
				\begin{subfigure}{0.35\textwidth}
					\includegraphics[width=\linewidth]{vectorizationImage.png}
				\end{subfigure}\hfil % <-- added
				\begin{subfigure}{0.23\textwidth}
					\includegraphics[width=\linewidth]{betti-curve.png}
				\end{subfigure}\hfil % <-- added
				\begin{subfigure}{0.35\textwidth}
					\includegraphics[width=\linewidth]{persistent_landscape.png}
				\end{subfigure}
			\end{figure}
		\end{frame}
	\section{Алгоритм классификации}
		\subsection{Алгоритм построения векторного представления}
		\begin{frame}
			\frametitle{Общая идея алгоритма}
			\begin{itemize}
				\item По изображению строим фильтрацию;
				\item По построенной фильтрации находим кубический комплекс, персистентные гомологии и строим диаграмму устойчивости;
				\item Векторизуем диаграмму устойчивости, получаем векторное представление, которое можно использовать в моделях машинного обучения.
			\end{itemize}
		\end{frame}
		\begin{frame}
			\frametitle{Пайплайн используемого метода построения векторного представления}
			Построить фильтрацию для изображения можно различными методами. В настоящей работе было построено 20 разнообразных фильтрацией, для каждой из которых были получены $0$ и $1$ персистентные гомологии, из диаграмм которых было получено 14 признаков. Таким образом, для одной картинки было получено $20 \times 2 \times 14 = 560$ признаков.
			\begin{figure}
				\centering
				\includegraphics[width=\linewidth]{pipelineDiagram.png}
			\end{figure}
		\end{frame}
		\begin{frame}
			\frametitle{Процесс получения векторного представления}
			\begin{figure}
				\centering
				\includegraphics[width=\linewidth]{pipe.jpg}
			\end{figure}
		\end{frame}
		\subsection{Полученные результаты}
		\begin{frame}
			\frametitle{Сравнение результатов различных методов машинного обучения}
			\begin{center}
				\begin{table}[!htbp]
					\centering
					\small
					\begin{tabularx}{\linewidth}{|X|X|X|}
						\hline
						Название модели & Значение на тренировочной выборке & Значение на тестовой выборке\\ \hline
						Логистическая регрессия & 0.989 & 0.903 \\
						\hline 
						Метод опорных векторов & 1.0 & 0.893 \\
						\hline
						Случайный лес & 1.0 & 0.89 \\
						\hline
						CatBoost & 1.0 & 0.893 \\ 
						\hline
					\end{tabularx}
					\caption{Значения базовых моделей на тренировочной и тестовой выборках}	
					\label{tabl:baselines}
				\end{table}
			\end{center}
		\end{frame}
		\begin{frame}
			\frametitle{Сравнение результатов различных методов машинного обучения}
			\begin{center}
				\begin{table}[!htbp]
					\centering
					\small
					\begin{tabularx}{\linewidth}{|X|X|X|}
						\hline
						Название модели & Значение на тренировочной выборке & Значение на тестовой выборке\\ \hline
						Логистическая регрессия & 0.911 & 0.917 \\
						\hline 
						Метод опорных векторов & 0.903 & 0.893 \\
						\hline
						Случайный лес & 0.89 & 0.89 \\
						\hline
						CatBoost & 0.905 & 0.9 \\ 
						\hline
					\end{tabularx}
					\caption{Значения наилучших моделей, полученных в результате подбора параметров поиском по сетке, на тренировочной и тестовой выборках}	
					\label{tabl:gridsearch}
				\end{table}
			\end{center}
		\end{frame}
		\begin{frame}
			\frametitle{Подбор гиперпараметров и отбор признаков}
			\begin{figure}[!htbp]
				\centering % <-- added
				\begin{subfigure}{0.25\textwidth}
					\includegraphics[width=\linewidth]{log_diff_features_train.png}
					\caption{Логистическая регрессия на тренировочной выборке}
					\label{fig:1}
				\end{subfigure}\hfil % <-- added
				\begin{subfigure}{0.25\textwidth}
					\includegraphics[width=\linewidth]{svm_diff_features_train.png}
					\caption{Метод опорных векторов на тренировочной выборке}
					\label{fig:2}
				\end{subfigure}\hfil % <-- added
				\begin{subfigure}{0.25\textwidth}
					\includegraphics[width=\linewidth]{rf_diff_features_train.png}
					\caption{Случайный лес на тренировочной выборке}
					\label{fig:3}
				\end{subfigure}
				
				\begin{subfigure}{0.25\textwidth}
					\includegraphics[width=\linewidth]{log_diff_features_test.png}
					\caption{Логистическая регрессия на тестовой выборке}
					\label{fig:4}
				\end{subfigure}\hfil % <-- added
				\begin{subfigure}{0.25\textwidth}
					\includegraphics[width=\linewidth]{svm_diff_features_test.png}
					\caption{Метод опорных векторов на тестовой выборке}
					\label{fig:5}
				\end{subfigure}\hfil % <-- added
				\begin{subfigure}{0.25\textwidth}
					\includegraphics[width=\linewidth]{rf_diff_features_test.png}
					\caption{Случайный лес на тестовой выборке}
					\label{fig:6}
				\end{subfigure}
				\caption{Сравнение различных моделей, которые используют различные признаки}
				\label{accuracies_diff_features}
			\end{figure}
		\end{frame}
		\begin{frame}
			\frametitle{Первые несколько изображений, на которых ошибся классификатор}
			\begin{figure}[!htbp]
				\begin{center}
					\includegraphics[width=0.5\textwidth]{missclassified.png}\\
					%\caption{Первые несколько изображений, на которых ошибся классификатор}
					\label{missclassified}
				\end{center}
			\end{figure}
		\end{frame}
		
	\section{Заключение}
		\subsection{Результаты}
		\begin{frame}
			\frametitle{Результаты}
			\begin{itemize}
				\item В ходе данной работы был изучен теоретический материал по алгебраической и прикладной топологии и машинному обучению. 
				\item Был реализован алгоритм классификации датасета MNIST, используя только топологические свойства изображенных рукописных цифр, который выдавал высокий уровень точности.
				
				\item В ходе данной работы было выявлено, что реализованный алгоритм является эффективным алгоритмом понижения размерности пространства признаков. 
				\item Было проведено сравнение с другими моделями машинного обучения, в результате которого было обнаружено, что реализованный алгоритм показывает более высокую точность классификации при меньшем числе признаков.
			\end{itemize}
		\end{frame}
		\begin{frame}
			\centering
			
			\begin{beamercolorbox}[sep=8pt,center,colsep=-4bp,rounded=true,shadow=true]{institute}
				\usebeamerfont{institute}\insertinstitute
			\end{beamercolorbox}
			
			{\usebeamercolor[fg]{titlegraphic}\inserttitlegraphic\par}
			
			\begin{beamercolorbox}[sep=8pt,center,colsep=-4bp,rounded=true,shadow=true]{title}
				\usebeamerfont{title}\inserttitle\par%
				\ifx\insertsubtitle\@empty%
				\else%
				\vskip0.25em%
				{\usebeamerfont{subtitle}\usebeamercolor[fg]{subtitle}\insertsubtitle\par}%
				\fi%     
			\end{beamercolorbox}%
			
			\vskip1em\par
			
			\begin{beamercolorbox}[sep=8pt,center,colsep=-4bp,rounded=true,shadow=true]{date}
				\usebeamerfont{date}\insertdate
			\end{beamercolorbox}\vskip0.5em
			
			\begin{beamercolorbox}[sep=8pt,center,colsep=-4bp,rounded=true,shadow=true]{author}
				\usebeamerfont{author}\insertauthor
			\end{beamercolorbox}
			
			
		\end{frame}
\end{document}