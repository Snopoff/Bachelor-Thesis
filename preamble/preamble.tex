\documentclass[14pt,a4paper]{report}

\usepackage[14pt]{extsizes}

%%% Работа с русским языком
\usepackage{cmap}					% поиск в PDF
\usepackage{mathtext} 				% русские буквы в фомулах
\usepackage[T2A]{fontenc}			% кодировка
\usepackage[utf8]{inputenc}			% кодировка исходного текста
\usepackage[english,russian]{babel}	% локализация и переносы
%\usepackage[style=gost-numeric]{biblatex}
%\bibliographystyle{ugost2008.bst}

%%% Реализация библиографии пакетами biblatex и biblatex-gost с использованием движка biber %%%

\usepackage{csquotes} % biblatex рекомендует его подключать. Пакет для оформления сложных блоков цитирования.
	\usepackage[%
backend=biber,% движок
bibencoding=utf8,% кодировка bib файла
sorting=none,% настройка сортировки списка литературы
style=gost-numeric,% стиль цитирования и библиографии (по ГОСТ)
language=autobib,% получение языка из babel/polyglossia, default: autobib % если ставить autocite или auto, то цитаты в тексте с указанием страницы, получат указание страницы на языке оригинала
autolang=other,% многоязычная библиография
clearlang=true,% внутренний сброс поля language, если он совпадает с языком из babel/polyglossia
sortcites=true,% сортировать номера затекстовых ссылок при цитировании (если в квадратных скобках несколько ссылок, то отображаться будут отсортированно, а не абы как)
doi=false,% Показывать или нет ссылки на DOI
isbn=false,% Показывать или нет ISBN, ISSN, ISRN
eprint=false
]{biblatex}[2016/09/17]
\renewcommand*{\mkbibhdnamefamily}[1]{#1}
\renewcommand*{\multicitedelim}{\addcomma\space}
\renewcommand*{\newblockpunct}{\addperiod\addnbspace\cyrdash\space\bibsentence}
\DeclareSourcemap{
	\maps{
	\map[overwrite]{ % переделка формата записи даты
	\step[fieldsource=urldate,
	match=\regexp{([0-9]{2})\.([0-9]{2})\.([0-9]{4})},
	replace={$3-$2-$1$4}, % $4 вставлен исключительно ради нормальной работы программ подсветки синтаксиса, которые некорректно обрабатывают $ в таких конструкциях
	final]
}}}
%%%END

\addbibresource{outro/resources.bib}

\usepackage{lipsum}
\usepackage{etoolbox}
\usepackage{color}

%%% Работа с картинками
\usepackage{graphicx}  % Для вставки рисунков
\graphicspath{{images/}}  % папки с картинками
\setlength\fboxsep{3pt} % Отступ рамки \fbox{} от рисунка
\setlength\fboxrule{1pt} % Толщина линий рамки \fbox{}
\usepackage{wrapfig} % Обтекание рисунков и таблиц текстом
\usepackage[export]{adjustbox}
\usepackage{subcaption}
\usepackage{float}

%%% Дополнительная работа с математикой
\usepackage{amsmath,amsfonts,amssymb,amsthm,mathtools} % AMS
\usepackage{icomma} % "Умная" запятая: $0,2$ --- число, $0, 2$ --- перечисление
\usepackage{systeme}

\usepackage{tabularx}
\usepackage{tikz-cd}
\usetikzlibrary{babel}
%Adds "References" to the table of contents
\usepackage[nottoc]{tocbibind} 
\usepackage{tocloft}
\renewcommand\cftchapfont{\mdseries}
\renewcommand\cftchappagefont{\mdseries}
% abs and norm 
\DeclarePairedDelimiter\abs{\lvert}{\rvert}%
\DeclarePairedDelimiter\norm{\lVert}{\rVert}%

% Swap the definition of \abs* and \norm*, so that \abs
% and \norm resizes the size of the brackets, and the 
% starred version does not.
\makeatletter
\let\oldabs\abs
\def\abs{\@ifstar{\oldabs}{\oldabs*}}
%
\let\oldnorm\norm
\def\norm{\@ifstar{\oldnorm}{\oldnorm*}}
\makeatother

%Algorithms
%http://blog.harrix.org/article/648
\usepackage[ruled,vlined]{algorithm2e}

\SetKwInput{KwData}{Исходные параметры}
\SetKwInput{KwResult}{Результат}
\SetKwInput{KwIn}{Входные данные}
\SetKwInput{KwOut}{Выходные данные}
\SetKwIF{If}{ElseIf}{Else}{если}{тогда}{иначе если}{иначе}{конец условия}
\SetKwFor{While}{до тех пор, пока}{выполнять}{конец цикла}
\SetKw{KwTo}{от}
\SetKw{KwRet}{возвратить}
\SetKw{Return}{возвратить}
\SetKwBlock{Begin}{начало блока}{конец блока}
\SetKwSwitch{Switch}{Case}{Other}{Проверить значение}{и выполнить}{вариант}{в противном случае}{конец варианта}{конец проверки значений}
\SetKwFor{For}{цикл}{выполнять}{конец цикла}
\SetKwFor{ForEach}{для каждого}{выполнять}{конец цикла}
\SetKwRepeat{Repeat}{повторять}{до тех пор, пока}
\SetAlgorithmName{Алгоритм}{алгоритм}{Список алгоритмов}

% hyper refs
\usepackage{hyperref}

%\parindent=12.5mm
\setlength{\parindent}{12.5mm}
\usepackage{indentfirst}
\linespread{1.5}

\usepackage[left=3cm,right=1.5cm,
top=2cm,bottom=2cm]{geometry}

\newcommand{\changefont}{%
	\fontsize{14}{16}\selectfont
}

%\usepackage{fancyhdr}
%\fancypagestyle{plain}{%
%	\fancyhf{} % clear all header and footer fields
%	\fancyhead[C]{\changefont\thepage} % except the center
%}
%\renewcommand{\headrulewidth}{0pt}

\AtBeginDocument{\renewpagestyle{plain}{\sethead{}{\thepage}{}}}

\pagestyle{myheadings}
\makeatletter
\renewcommand{\@oddfoot}{}
\renewcommand{\@oddhead}{\hfil\thepage\hfil}
\renewcommand{\@evenfoot}{}
\renewcommand{\@evenhead}{\hfil\thepage\hfil}
\makeatother

\newtheorem{definition}{Определение}
\newtheorem*{theorem*}{Теорема}
\newtheorem{theorem}{Теорема}
\newtheorem{lemma}{Лемма}
\newtheorem*{lemma*}{Лемма}
\newtheorem{consequence}{Следствие}


\newcommand{\Om}[1][]{$\Omega_{#1}$}
\newcommand{\diam}{\mathrm{diam}}
\newcommand{\cat}[1]{{\normalfont\textbf{#1}}}
\newcommand{\Cech}{\mathrm{Cech}}
\newcommand{\Rips}{\mathrm{Rips}}

\DeclareMathOperator{\im}{im}
\newcommand{\quotient}[2]{{\raisebox{.2em}{$#1$}\left/\raisebox{-.2em}{$#2$}\right.}}

\usepackage{titlesec}

\newcommand{\chapfnt}{\fontsize{16}{19}}
\newcommand{\secfnt}{\fontsize{14}{17}}
\newcommand{\ssecfnt}{\fontsize{14}{14}}

\titleformat{\chapter}%[display]
{\normalfont\chapfnt\bfseries\filcenter}{\thechapter}{24pt}{\chapfnt}
%{\normalfont\chapfnt\bfseries}{\chaptertitlename\ \thechapter}{20pt}{\chapfnt}

\titleformat{\section}
{\normalfont\bfseries\filcenter}{\thesection}{1em}{}
\titleformat{\subsection}
{\normalfont\bfseries\filcenter}{\thesubsection}{1em}{}

\titlespacing*{\chapter} {0pt}{50pt}{40pt}
\titlespacing*{\section} {0pt}{3.5ex plus 1ex minus .2ex}{2.3ex plus .2ex}
\titlespacing*{\subsection} {0pt}{3.25ex plus 1ex minus .2ex}{1.5ex plus .2ex}

\newcommand\restr[2]{{% we make the whole thing an ordinary symbol
		\left.\kern-\nulldelimiterspace % automatically resize the bar with \right
		#1 % the function
		\vphantom{\big|} % pretend it's a little taller at normal size
		\right|_{#2} % this is the delimiter
}}

\usepackage{setspace}
\usepackage{listings}
\usepackage[toc, page]{appendix}
\renewcommand{\appendixname}{Приложение}
\renewcommand{\appendixpagename}{Приложение}
\renewcommand{\appendixtocname}{Приложение}
\usepackage{listingsutf8}
\definecolor{mygreen}{RGB}{28,172,0} % color values Red, Green, Blue
\definecolor{mylilas}{RGB}{170,55,241}
\lstset{language=python,%
	basicstyle=\linespread{0.8}\ttfamily\footnotesize, %
	breaklines=true,%
	morekeywords={matlab2tikz},
	extendedchars=\true,
	keywordstyle=\color{blue},%
	morekeywords=[2]{1}, keywordstyle=[2]{\color{black}},
	identifierstyle=\color{black},%
	stringstyle=\color{mylilas},
	commentstyle=\color{mygreen},%
	showstringspaces=false,%without this there will be a symbol in the places where there is a space
	numbers=left,%
	numberstyle={\tiny \color{black}},% size of the numbers
	numbersep=5pt, % this defines how far the numbers are from the text
	emph=[1]{for,end,break},emphstyle=[1]\color{red}, %some words to emphasise
}